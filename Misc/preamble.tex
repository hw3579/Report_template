\usepackage[utf8]{inputenc}
%\usepackage[English]{inputenc}
\usepackage{amsmath}   % Package for math files
\usepackage{txfonts}
\usepackage{parskip}   % No indent, but instead paragraphs
\usepackage{graphicx}  % Place figures
\usepackage{caption}   % Place captions in tables and figures
\usepackage{subcaption}% 
\usepackage{subfiles}  % 
\usepackage[T1]{fontenc} 
\usepackage[euler]{textgreek} % To get greek letters as we know them
\usepackage{amssymb}   % 
\usepackage{placeins}  % \FloatBarrier so figures can't float beyond some point in text
\usepackage{fullpage}  % Uses more of the page
\usepackage{float}     % Able to make figures and tables float \begin{figure}[H] to keep them HERE
\usepackage[version=4]{mhchem} % \ce{} to write chemical eq.
\usepackage{siunitx}   % Ex: \si{\meter\per\square\second}
\usepackage{booktabs}  % Behind-the-scenes optimization of tables. \toprule, \midrule, \bottomrule
\usepackage{hyperref}  % Ability to click on references like equations, figures, sections etc. \ref{eq:my_eq} clickable
\hypersetup{
    colorlinks,
    citecolor=black,
    filecolor=black,
    linkcolor=black,
    urlcolor=black
}
\usepackage[autolinebreaks,useliterate,numbered]{mcode} % Ability to paste smooth MATLAB code

\newcommand{\figref}[1]{\figurename~\ref{#1}} %Nice reference to figures