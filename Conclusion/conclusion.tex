\section{Conclusion}
\label{sec:conclusion}
%The conclusion should, as the abstract, wrap up what you have found in the experiment. It should state what you have done and what you have found. The conclusion should only state what is obvious from the discussion in section \ref{sec:results}; no new information should arise here. The conclusion is the first place the reader will go looking if he wants to get an overview of the report. If it is interesting, he might read the rest. Be sure that the conclusion is short and concise, but do not omit important information. You have one shot at presenting your results: if you have done excellent work at the lab it doesn't matter if you are unable to present the results in an appealing way. The report is your only way of communicating and presenting your hard work.


From the results of the regression analysis, 
the R-square value of both experiment A1 and A2 are larger than 0.98, 
which demonstrates a strong linear correlation between the x and y axes.
Additionally, from the above it follows that $C_v = 0.955$,$C_d=0.615$(3mm),$C_d=0.8816$
(6mm). Both of the coefficient are less than 1.This satisfies the hypothetical conditions.

Therefore, the data from the regression analysis is valid.

For $C_v$, due to vena contracta , the real area that approximates the area 
of the orifice with a little smaller, 
and its value of 0.95 is consistent with the reality.

For $C_d$(3mm), the value is 0.615, whereas it should actually be between 0.8 and $C_v$, 
probably due to experimental error. 
This should be done several times for the 3mm diameter orifice to avoid 
experimental coincidence and make it more accurate.

For $C_d$(6mm), the value is 0.8816, which is less than $C_v$ and indicate the 
$C_c < 1 $\eqref{13}.This data is inaccurate but should not be significant.